\chapter{Conclusion}

Urban centers have long-since been fighting the natural occurrence of traffic congestion in the past decades. To help solve this problem, several traffic models were already developed while also considering weather as an effect. However, these models were not designed and tested in places with substandard drainage systems and poor road management. This research aims to adapt a weather-aware traffic model to cater these places. In this paper, analysis on the patterns of traffic and weather data were explored through correlation analysis. Features were also extracted to gain more insights as to what variables have a relationship with traffic congestion. An existing traffic model using DBNs and data fusion techniques was adapted to predict the traffic congestion in selected roads in Manila. Different experiment settings were done to show how different traffic and weather features affect the traffic prediction. 

Analysis show that patterns and trends are still evident even in urban centers where disruptions in traffic and weather are present. Additionally, extracting information from the categorical data of traffic congestion is possible and can be quite insightful. In evaluating the model in predicting future traffic, results show that using the immediate past traffic is insightful and bears importance in decision making. Results also show that fusing at decision-level improves accuracy in prediction better than feature-level. Moreover, though DBNs bear the ability to extract patterns, RNNs outperforms DBN in extracting the patterns of time-series data.
