\chapter{Conclusion}

Urban centers have long-since been fighting the natural occurrence of traffic congestion in the past decades. To help solve this problem, several traffic models were already developed while also considering weather as an effect. However, these models were not designed and tested in places with substandard drainage systems and poor road management. This research aims to develop a traffic model that considers data on traffic condition and weather variables such as wind speed, wind gust, temperature, humidity, dew point, precipitation, visibility, pressure, cloud cover, heat index, and feels-like.

The relationship between traffic and weather were analyzed through exploratory data analysis. First, the seasonality for weather and traffic was first explored and defined. The weather seasons such as wet and dry seasons are also defined using precipitation analysis. Disruptions in the normal pattern of these variables were also identified. Then, the correlation between traffic and weather were explored through correlating between weather variables, connected road segments, and the immediate past traffic and defined traffic trends. Results of the exploratory data analysis show that precipitation is a factor that disrupts the normal pattern of traffic after comparing between the traffic a week before the recorded landfall of a typhoon, and the week that that typhoon was present. Moreover, correlating traffic weeks ago with the current traffic shows that the general traffic 6 weeks before is more similar to other weeks for both regular days and disrupted days in which typhoon Goring was present. Furthermore, analysis show that it is important that the fact that a day is a working day and an hour is a peak hour is considered when dealing with a time sensitive data. It is also observed there is an intensity relationship between roads such that the traffic condition in one road is consistent with its connecting road based on the peak traffic hours, but differs in intensities as the road gets farther from the main road. 

Afterwards, the study implemented two prediction models, Traffic-Only Model (TOM) and Weather-Only Model (WOM), using Deep Belief Network (DBN) to predict traffic condition intensity based from traffic and weather data, respectively. The findings from the exploratory data analysis were used to select and engineer features, using them in the model to better predict the current traffic condition. The final features selected and engineered as as follows:

\begin{enumerate}
\item Temporal Information of the respective traffic record represented as Month, Day, Hour, Minute, Day of Week,
\item Traffic a Day before represented as L, ML, M, MH, H
\item Traffic 6 weeks ago represented as L, ML, M, MH, H
\item Current Traffic represented as L, ML, M, MH, H
\item Rolling and Expanding Traffic Features (mean, max, and minimum) for windows 4, 8, 24, 48, and 96 represented as L, ML, M, MH, H
\item Flags for Working day and Peak Hour represented as 0 and 1
\item Weather Variables (wind speed, wind gust, temperature, humidity, dew point, precipitation, visibility, pressure, cloud cover, heat index, and feels-like) represented in their respective measurements 
\end{enumerate}

Different combinations of these features for their respective prediction models (TOM and WOM) were made to further evaluate which features best represent traffic. Results of the experimentation show that the model best predicts traffic on normal days. Moreover, the model best predicts the \textit{moderate}and \textit{light} traffic condition intensity because there are more reports on these traffic conditions than the \textit{heavy} traffic condition that may have captured sudden peaks and disruptions in trafic. 

Different fusion approaches, such as feature fusion and decision fusion, were also tested. Moreover, different data fusion algorithms in the decision level,  were also compared. These data fusion algorithms were DBN, Recurrent Neural Network (RNN), and Weighted Average (WA) using Least Square Estimate. Predictions were evaluated with the different fusion approaches and algorithms. Results show that fusing at the feature level does not differ significantly than fusing at the decision level. Moreover, fusing at the decision level using RNN outperforms fusing with WA and DBN. The capability of LSTM of RNN contributes to the high performance of the model, thus outperforming the other algorithms. 

The sensitivity of the models TOM and the fusion model implemented in DBN were evaluated. Analysis shows that the flags for working day and peak hours were connected with the temporal information of the data. Moreover, information on past immediate traffic such as the mean traffic of a 6 weeks ago, traffic a day ago, rolling and expanding features an hour and a day ago, highly affects the prediction of the model. Additionally, in analyzing the fusion model, the inclusion of weather as a factor in predicting traffic only contributed 27\% in the prediction. 

