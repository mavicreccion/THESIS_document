\chapter{Conclusion}

Urban centers have long-since been fighting the natural occurrence of traffic congestion in the past decades. To help solve this problem, several traffic models were already developed while also considering weather as an effect. However, these models were not designed and tested in places with substandard drainage systems and poor road management. This research aims to develop a traffic model that considers data on traffic condition and weather variables such as wind speed, wind gust, temperature, humidity, dew point, precipitation, visibility, pressure, cloud cover, heat index, and feels-like.

The relationship between traffic and weather were analyzed through exploratory data analysis. First, the seasonality for weather and traffic was first explored and defined. The weather seasons such as wet and dry seasons are also defined using precipitation analysis. Then, the correlation between traffic and weather were explored through correlating between weather variables, connected road segments, and the immediate past traffic and defined traffic trends. Results show that the current traffic is correlated with its past traffic 1 week ago. Moreover, data shows that majority of the traffic condition during working days and non-working days consist of \textit{light} to \textit{moderate} traffic conditions. It is also observed there is a relationship between roads in terms of traffic condition intensity such that the traffic condition in one road is consistent with its connecting road based on the peak traffic hours, but differs in intensities as the road gets farther from the main road. 

As for analyzing the weather, findings show that weather has a weak correlation with traffic. Moreover, among all weather variables, those that have the strongest correlation with traffic, only describes the transition from dawn to morning, or simply the beginning of the morning peak hour, and not its effect on traffic. Findings show that there is no derivable relationship between weather variables and traffic because of the limited representation of traffic data. 

Disruptions in the normal pattern of these variables were also identified. Disruptions are defined as the instances where rain is continuously present for 7 hours from 0:00 to 21:00 during working days. Majority of the found disrupted days are present in the wet season. In analyzing disruptions, it was discovered that traffic that traffic is less evident during wet season where precipitation is present that can base disruptions in the traffic. Disruptions may be disregarded if only traffic a week in the past is considered. Thus, traffic weeks ago is observed, and findings show that current traffic is correlated with the traffic in the past 6 weeks for both normal and disrupted periods. 

Afterwards, the study implemented two prediction models, Traffic-Only Model (TOM) and Weather-Only Model (WOM), using Deep Belief Network (DBN) to predict traffic condition intensity based from traffic and weather data, respectively. The findings from the exploratory data analysis were used to select and engineer features, using them in the model to better predict the current traffic condition. The final features selected and engineered are as follows:

\begin{enumerate}
\item Temporal Information of the respective traffic record represented as Month, Day, Hour, Minute, Day of Week;
\item Traffic a Day before represented as L, ML, M, MH, H;
\item Traffic 6 weeks ago represented as L, ML, M, MH, H;
\item Current Traffic represented as L, ML, M, MH, H;
\item Rolling and Expanding Traffic Features (mean, max, and minimum) for windows 4, 8, 24, 48, and 96 represented as L, ML, M, MH, H;
\item Flags for Working day and Peak Hour represented as 0 and 1; and,
\item Weather Variables (wind speed, wind gust, temperature, humidity, dew point, precipitation, visibility, pressure, cloud cover, heat index, and feels-like) represented in their respective measurements 
\end{enumerate}

Different combinations of these features for their respective prediction models (TOM and WOM) were made to further evaluate which features best represent traffic. Results of the experimentation show that the model best predicts traffic on normal days, and road segments with less diverse traffic. In predicting for disrupted days, TOM cannot accurately predict the abrupt transitions of traffic conditions. Predictions of TOM greatly improves after including information on immediate past traffic, such as rolling and expanding window features with small windows. On the other hand, WOM couuld only predict normal traffic trends, and traffic conditions often present in the trend such as traffic conditions of \textit{light} to \textit{moderate}. Using only weather variables as factors in predicting traffic could offer a contributing weight in predicting traffic. However, weather variables alone cannot be used to predict traffic, because of its weak correlation with traffic. 

Different fusion approaches, such as feature fusion and decision fusion, were also tested. Moreover, different data fusion algorithms in the decision level were also compared. These data fusion algorithms were DBN, Recurrent Neural Network (RNN), and Weighted Average (WA) using Least Square Estimate. Predictions were evaluated with the different fusion approaches and algorithms. Results show that fusing traffic and weather at the decision level generates a better prediction than fusing at the feature level. Moreover, fusing at the decision level using RNN outperforms fusing with WA and DBN. The capability of LSTM of RNN contributes to the high performance of the model, thus outperforming the other algorithms. 

The sensitivity of the models TOM and the fusion model implemented in DBN were evaluated. Analysis shows that the flags for working day and peak hours were connected with the temporal information of the data. Moreover, information on past immediate traffic such as the mean traffic of a 6 weeks ago, traffic a day ago, rolling and expanding features an hour and a day ago, highly affects the prediction of the model. Additionally, in analyzing the fusion model, the inclusion of weather as a factor in predicting traffic only contributed 27\% in the prediction. 

