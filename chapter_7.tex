\chapter{Future Works}

%%%% REAL TIME DATA COLLECTION %%%%
Using historical data is a good and common entry for data-driven researches. However, in the case of traffic, it is better to consider real-time information to predict future traffic conditions. In the event that something unexpected happens, such as road accidents or just-closed lanes, the historical traffic data may not be as useful as it is. Sudden events may also include typhoon warnings, which has no annual pattern and cannot be be easily predicted. With historical traffic and weather data combined with real-time information, future traffic predictions may become more accurate.

%%%% SOCIAL MEDIA REPORTS %%%%
Considering traffic reports in social media may also be an additional feature in the model. Social media platforms such as Twitter and Facebook allows people to let out their frustrations including everyday problems such as road traffic. One study \shortcite{gu2016twitter} retrieves real-time traffic information by crawling, processing, and filtering public tweets to detect road traffic incidents using natural processing language techniques. Using this as an added feature in the model may boost its accuracy. 

%%%% ROAD CATASTROPHES (accidents, constructions, closed) %%%%
Unpredicted road catastrophes also contribute to traffic congestion, including road accidents, road constructions, and recently-closed lanes. Car accidents block the roads, which then hinder other vehicles to efficiently pass through, causing heavy traffic jam \shortcite{wang2009impact}. Road constructions such as building of bridges, expanding the roads, fixing the water pipes underneath the roads, among others, also tend to increase traffic congestion, especially if there is no notice beforehand. The sudden closing of a road may also contribute to traffic jams, since it disrupts the normal flow of traffic. Studying these kinds of road catastrophes may be helpful in achieving a better traffic prediction. Data may be collected as historical data or using real-time information.

%%%% FLOOD %%%%
In urban places where drainage systems are not properly managed or road infrastructures are not properly built and maintained, the amount of rain accumulating on the roads may increase traffic congestion as vehicles have difficulty moving forward. It is worse on places where adverse weather conditions such as typhoons bring heavy rains, which then causes immense flooding if the rain is not properly drained. Although the existing application Waze allows drivers and passengers to report flooding incidents in the roads, Waze does not use these reports in learning and predicting future traffic conditions. In future works, considering the height of flood as a contributing factor may help the model in predicting future traffic condition. 
