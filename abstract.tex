%%%%%%%%%%%%%%%%%%%%%%%%%%%%%%%%%%%%%%%%%%%%%%%%%%%%%%%%%%%%%%%%%%%%%%%%%%%%%%%%%%%%%%%%%%%%%%%%%%%%%%
%
%   Filename    : abstract.tex 
%
%   Description : This file will contain your abstract.
%                 
%%%%%%%%%%%%%%%%%%%%%%%%%%%%%%%%%%%%%%%%%%%%%%%%%%%%%%%%%%%%%%%%%%%%%%%%%%%%%%%%%%%%%%%%%%%%%%%%%%%%%%

\begin{abstract}
Numerous traffic models have been developed to alleviate urban traffic congestion, while also considering weather variables such as temperature, wind gust, weather condition, and rainfall. However, while these traffic models do predict traffic congestion and have produced high accuracies, these models were designed and evaluated in the setting of industrialized countries. In countries that have inadequate drainage systems and poor road infrastructure, rainfall does not runoff effectively and causes immense flooding in prolonged rain periods. Hence, flooding has proven to have a substantial impact on road infrastructure as extreme and prolonged rains become a regular occurrence. As such, this research proposes a traffic model that incorporates the aforementioned weather variables and the consequential flooding in Manila. Weather and flood data will be collected and correlated with one another and with traffic condition to identify factors that the model will be trained for. The research will adapt a weather-aware traffic model with the use of Deep Belief Networks and data fusion techniques. Then, the performance of the model with different setups, such as the different fusion levels of the whole model, the fusion techniques for the decision and feature fusion and the individual predictors, will be evaluated using the root mean squared error (RMSE), mean absolute percentage error (MAPE), and sensitivity analysis. 

\begin{flushleft}
\begin{tabular}{lp{4.25in}}
\hspace{-0.5em}\textbf{Keywords:}\hspace{0.25em} & Traffic modeling, weather information, deep learning, data fusion\\
\end{tabular}
\end{flushleft}
\end{abstract}
