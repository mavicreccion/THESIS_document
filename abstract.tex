%%%%%%%%%%%%%%%%%%%%%%%%%%%%%%%%%%%%%%%%%%%%%%%%%%%%%%%%%%%%%%%%%%%%%%%%%%%%%%%%%%%%%%%%%%%%%%%%%%%%%%
%
%   Filename    : abstract.tex 
%
%   Description : This file will contain your abstract.
%                 
%%%%%%%%%%%%%%%%%%%%%%%%%%%%%%%%%%%%%%%%%%%%%%%%%%%%%%%%%%%%%%%%%%%%%%%%%%%%%%%%%%%%%%%%%%%%%%%%%%%%%%

\begin{abstract}
Several traffic models were already developed to alleviate traffic congestion, while also considering the effect of weather. However, they were not designed and tested in countries with substandard drainage systems, low traffic infrastructure investments, and underdeveloped roads. If we look at these places, there have been reports of other weather variables and their resulting natural disasters, having some effect on the traffic condition. Differences in public infrastructures have caused the continuous accumulation of rainfall to lead to the slowing down of cars, thereby increasing traffic congestion. As such, this research proposes a traffic model that incorporates the aforementioned weather variables present in Manila, and other features that describe the trends in traffic and weather for both normal and disrupted periods, to predict the current traffic condition intensity. From the data used, it was observed that traffic consists mostly of instances of \textit{light} and \textit{moderate} traffic. It was also observed that weather, especially an abundance of precipitation, disrupts the normal trend of traffic, and that weather has a weak correlation with traffic. From such findings, features that will help predict traffic were selected and engineered. Variables such as temporal information of traffic, previous and current traffic conditions, rolling and expanding features, flags for working day and peak hour, and weather variables were the final selection of features that was used for the model. Models based on traffic, weather, and a fusion of the two at feature and decision levels were implemented using DBN. Other fusion algorithms such as RNN, and WA using Least Square Estimate were also implemented for comparison. The various combinations of features, the different fusion approaches and algorithms, and the sensitivity of the models were evaluated. Results show that the traffic-based prediction outperforms weather-based traffic prediction. Moreover, the model that fuses in the decision level performs better than fusing in the fusion level. After implementing various fusion algorithms, it was found that RNN, especially at decision level, outperforms the other fusion algorithms.


\begin{flushleft}
\begin{tabular}{lp{4.25in}}
\hspace{-0.5em}\textbf{Keywords:}\hspace{0.25em} & Traffic modeling, weather information, deep learning, data fusion\\
\end{tabular}
\end{flushleft}
\end{abstract}
