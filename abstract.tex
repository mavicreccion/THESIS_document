%%%%%%%%%%%%%%%%%%%%%%%%%%%%%%%%%%%%%%%%%%%%%%%%%%%%%%%%%%%%%%%%%%%%%%%%%%%%%%%%%%%%%%%%%%%%%%%%%%%%%%
%
%   Filename    : abstract.tex 
%
%   Description : This file will contain your abstract.
%                 
%%%%%%%%%%%%%%%%%%%%%%%%%%%%%%%%%%%%%%%%%%%%%%%%%%%%%%%%%%%%%%%%%%%%%%%%%%%%%%%%%%%%%%%%%%%%%%%%%%%%%%

\begin{abstract}
Numerous traffic models have been developed to alleviate urban traffic congestion, while also considering weather variables such as temperature, wind gust, weather condition, and rainfall. However, while these traffic models do predict traffic congestion and have produced high accuracies, these models were designed and evaluated in the setting of industrialized countries. In countries that have inadequate drainage systems and poor road infrastructure, rainfall does not run off effectively and causes immense flooding in prolonged rain periods. Hence, flooding has proven to have a substantial impact on road infrastructure especially when extreme and prolonged rains become a regular occurrence. As such, this research proposes a traffic model that incorporates the aforementioned weather variables present in Manila, and other features that describes the trends in traffic and weather for both normal and disrupted periods, to predict the current traffic condition intensity. From the data used, it was observed that traffic consists mostly of instances of \textit{light} and \textit{moderate} traffic. It was also observed that weather, especially an abundance of precipitation, can disrupt the normal trend of traffic.It was also observed that weather has a weak correlation with traffic. From such findings, features that will help predict traffic were selected and engineered. Variables such as temporal information of traffic, previous and current traffic conditions, rolling and expanding features, flags for working day and peak hour, and weather variables were the final selection of features that was used for the model. Models based on traffic, weather, and a fusion of the two at feature and decision levels were implemented using DBN. Other fusion algorithms such as RNN, and WA using Least Square Estimate were also implemented for comparison. The various combinations of features, the different fusion approaches and algorithms, and the sensitivity of the models were evaluated. Evaluating the models shows that the traffic-based prediction outperforms weather-based traffic prediction. Moreover, the model that fuses in the decision level performs better than fusing in the fusion level. After implementing various fusion algorithms, it was found that RNN, especially at decision level, outperforms the other fusion algorithms.  


\begin{flushleft}
\begin{tabular}{lp{4.25in}}
\hspace{-0.5em}\textbf{Keywords:}\hspace{0.25em} & Traffic modeling, weather information, deep learning, data fusion\\
\end{tabular}
\end{flushleft}
\end{abstract}
